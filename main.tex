\documentclass{article} 

\usepackage{amsmath}

\usepackage[english, russian]{babel} 
\usepackage{array,tabularx,tabulary,booktabs}


\title{Log}
\author{} 
\date{\today}


\begin{document}

\section*{Домашнее задание 1}

ФКН НИУ ВШЭ 2023--2024 \vspace{1.5em}

Анжелика выбирает компанию, в которой она хочет работать, когда закончит университет; но она никак не может определиться с размером будущей компании: <<Рассматривать преимущественно небольшие стартапы или всё-таки мегакорпорации?>>

Вчера она прочитала в журнале <<The American Economic Review>>, что количество работников в компаниях может хорошо приближаться логнормальным распределением, т.е. таким распределением случайной величины, логарифм которой распределен нормально.

Анжелика просит вас помочь ей с выбором и написать Android приложение, которое будет генерировать очередное случайное число, подчиняющееся логнормальному закону распределения, и отображать его в текстовом поле. Параметры $\mu$ и $\sigma ^2$ она хочет задавать сама непосредственно в приложении.
Будьте внимательны, от вас зависит будущее Анжелики!

Интерфейс приложения должен содержать элементы, перечисленные в табл.~\ref{table:layout:ids}.


\begin{table}[h]
\begin{center}
    \begin{tabularx}{0.78\textwidth}{p{3ex} p{13ex} X p{10ex}}
\midrule\toprule
№ & View       & id            &   Описание \\ \midrule
0 & EditText        & mean\_val   &   $\mu$ \\
1 & EditText        & variance\_value        &  $\sigma ^2$  \\
2 & Button            & get\_random\_num  & $	 $   \\
3 & TextView        & random\_number\_result         & $	 $    \\

\bottomrule\midrule
    \end{tabularx}
    \caption{Элементы пользовательского интерфейса}
    \label{table:layout:ids}
\end{center}
\end{table}


При нажатии на кнопку с \texttt{@id/get\_random\_num} в текстовое поле TextView  (android.widget.TextView) с \texttt{@id/random\_number\_result} должна устанавливаться строка, 
содержащая сгенерированное число (число с плавающей точкой).
При повороте экрана телефона это число должно сохраняться.


Используйте develop ветку для разработки, основная ветка -- master/main. Настройте GitHub Actions для автоматической сборки и тестирования кода (unit- и UI-тесты), detekt для проверки качества кода.
Оформите релиз приложения в GitHub (раздел Releases), прикрепите APK файл и файл App Bundle к релизу, добавьте краткое описание и скриншоты приложения в README.md.


\end{document}